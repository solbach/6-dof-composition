\documentclass[10pt,a4paper]{scrartcl}
\usepackage[utf8]{inputenc}
\usepackage[german, english]{babel} 
\usepackage{graphicx}
\usepackage{amsmath}
\usepackage{amsfonts}
\usepackage{amssymb}
\usepackage{hyperref}
\usepackage{float}

% Kopf und Fußzeile
\usepackage{fancyhdr}
\usepackage{cite}

\author{Markus Solbach \\ \href{mailto:solbach@uni-koblenz.de}{\texttt{solbach@uni-koblenz.de}}}
\date{\today}


\pagestyle{fancy} %eigener Seitenstil
\fancyhf{} %alle Kopf- und Fußzeilenfelder bereinigen
\fancyhead[L]{\texttt{solbach@uni-koblenz.de}} %Kopfzeile links
\fancyhead[C]{} %zentrierte Kopfzeile
\fancyhead[R]{ 
\begin{tabular}{ll} 
Markus Solbach\\ 
\end{tabular}
} %Kopfzeile rechts
\renewcommand{\headrulewidth}{0.4pt} %obere Trennlinie

\fancyfoot[R]{\thepage} %Seitennummer
\renewcommand{\footrulewidth}{0.4pt} %untere Trennlinie


\begin{document}


\title{3D Composition (6 DOF)}
\subtitle{A Short overview}
\maketitle
\newpage

\section{Introduction}
Adding relative motion, given by odometry for instance (observation), to an absolute state its called composition. In our case we deal with a full 3 dimensional composition with 6 degrees of freedom (DOF). With other words it is possible to translate the system in $x, y, z$ direction what gives us the first 3 DOF and to rotate the system around $\phi, \theta, \psi$ (roll, pitch, yaw), that gives us the missing 3 DOF.
\\\\ In more detail. Let $X$ the state-vector and $Y$ the observation-vector. Defined as follows:
\begin{equation}
X = \begin{bmatrix} x_x \\ y_x \\ z_x \\ \phi_x \\ \theta_x \\ \psi_x \end{bmatrix}, \hspace{15 mm} Y = \begin{bmatrix} x_y \\ y_y \\ z_y \\ \phi_y \\ \theta_y \\ \psi_y \end{bmatrix}
\end{equation}


\section{Transformation}
% plus operator

\section{Jacobians}

%\begin{figure}[ht]
%	\centering
%  \includegraphics[width=.7\textwidth]{img/systemGraph2}
%	\caption{Modelled System with finite capacity region using \textit{JSIMgraph}}
%	\label{im:sys2}
%\end{figure}


\end{document}